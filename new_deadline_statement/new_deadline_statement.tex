%%%%%%%%%%%%%%%%%%%%%%%%%%%%%%%%%%%%%%%%%%%
%        New Deadline Statement Semantics
%%%%%%%%%%%%%%%%%%%%%%%%%%%%%%%%%%%%%%%%%%%
%      Author: Ben Lickly
%

\documentclass[12pt]{article}
\usepackage{listings}
\usepackage[usenames,dvipsnames]{color}
\begin{document}

\section{Introduction/Motivation}
This report is motivated by a desire to create C-level constructs
that compile down to deadline statements. One undesirable aspect of
that conversion is the limited number of deadline registers. This,
combined with the fact that we cannot achieve register spilling with
deadline registers, means that there is a limitation on the amount of
timing constructs that can be concurrently active in a program.
This would make timing constructs not compositional, in the sense
that two functions with timing constructs may be compilable and correct
individually, but composing them results in a program that cannot be compiled.
Clearly, this is undesirable.

\section{Main Idea}
The main idea is to use a single hardware timer register, and general purpose
registers in place of the individual deadline registers.
\begin{lstlisting}
	dead T, r2
	dead T, imm
\end{lstlisting}
becomes
\begin{lstlisting}
	dead r1, r2
	dead r1, imm
\end{lstlisting}
The semantics of the new deadline instruction is as follows:
First read the value of $r1$, and compare it to the current value of the hardware timer.  If the current timer value is before the value of $r1$, simply stall
until current timer value advances.  Once the current timer value reaches $r1$,
load the current timer value plus the value of $r2$ or $imm$ into $r1$.

\section{Caveats}

\subsection{deadbranch statements}
One of the problems with this approach is that it does not apply easily
to deadline statements that perform an action when their registers reach zero,
such as the deadbranch statement. One proposed solution would be to have
a single extra register that raises an interrupt handler when its deadline
is reached.  Deadbranch deadlines could then be implemented with software
keeping track of the next deadline with a priority queue and only having
the hardware wait on a single timer.
This additional software would add overhead when using deadbranch statements
and would also place a lower limit on how quickly two separate deadbranch
misses could be noticed, but would add the significant benefit of not
limiting the number of deadbranch statements that could be in flight at
a given time.

\subsection{Timer rollover}
One of the greatest benefits of having separate timing registers for each
timing block, is that each register is completely self-contained, and
refers exactly to the block that it is timing.
We can, for example, stop registers for counting when they get to zero,
which is impossible with a single global timing register.
The greatest difficulty that comes with having all deadline blocks synchronize
to a global timer is that of rollover. i.e. How can we assure that the change
of the global timer's most significant bit from 1 to 0 doesn't make deadlines
think that they have been missed?

One way to do this is to compare differences in time values rather than
the time values themselves.
For example, rather than checking for $t_1$ being earlier than $t_2$ by simply seeing if $t_1 < t_2$, we can check if $t_1 - t_2 < 0$.
In order for this comparison to be true, the values must of course be unsigned.
Notice that this results in sacrificing half of the range of the timers
in order to record which direction a time is.
This is because the new comparison operation only works when the difference
in timer values is within the range of an unsigned value,
whereas the simple comparison operation worked for any two timer values,
no matter how far apart they are from each other.

Small example:
Let us consider a small example with an 8-bit timer that wraps around.
% [-128,127]
Assume that we set a timer at time 100 to last for 50 time units.
In both cases, the deadline will be -106 (= 100 + 50 - 256).
Now let us assume that 25 time units later, at time 125, the next deadline is
hit. In order to know whether we should stall or not,
we need to know whether the time 125 is before time -106.
By the simplest check, we will say that 125 is after -106,
since $-106 < 125$.
By the difference check, however, we will check if time 125 is before -106
by checking if $125 - -106 < 0$.
This results in the check of $-25 < 0$ which is true.


\end{document}
